%preamble
\documentclass[12pt,a4paper]{article}

%inputs for firstpage
\def \CourseName {گزارش فاز دوم}
\def \Instructor {دکتر مسلم حبیبی}
\def \Semester {نیم‌سال دوم\\سال تحصیلی 99-00}


%packages
\usepackage[]{algorithm2e}
\usepackage{cite}
\usepackage{calc}
\usepackage{fancyhdr}
\usepackage{lipsum}
\usepackage{color}
\usepackage{ragged2e}
\usepackage[inline]{enumitem}
\usepackage[dvipsnames]{xcolor}
\usepackage{graphicx}
\usepackage{wrapfig}
\usepackage{float}
\usepackage[skip=12pt,indent=2em]{parskip}
\usepackage{setspace}
\usepackage{textcomp}
\usepackage{etoolbox}
\usepackage{xpatch}
\usepackage{tabu}
\usepackage{hyperref}


%for persian fonts
\usepackage{xepersian}
\defpersianfont\bnazanin{BNazanin}
\settextfont{BNazanin}

\title{
	\center
	\includegraphics[width=4cm, height=4cm]{images/shariflogo.jpg} 
	\small 
	\\ دانشگاه صنعتی شریف\\دانشکده مهندسی صنایع \\ درس \lr{MIS}
	\normalsize
	\\[60pt]
	\textbf{\huge\CourseName}
}
\author{
	\textbf{استاد درس:}
	\\
	\Instructor
 	\\
	\\
	\textbf{نام اعضای گروه:}
	\\مهدی محسنی 
	\\محراب کشاورز گیلده 
	\\احسان چشمی
	\\[45pt]
}
\date{{\small\Semester}}



%body
\begin{document}

\maketitle
\pagebreak
\tableofcontents
\pagebreak
\normalsize	
\section{\lr{Product Vision}} \label{section.productVision}


چشم انداز محصول توصیف کننده هدف یک محصول است که با چه قصدی آن را ایجاد کرده ایم و به چه اهدافی در رابطه با مشتریان خود و سایر افراد مرتبط با کسب و کار میخواهیم برسیم. چشم انداز محصول وضعیت آینده محصول را توصیف میکند و پتانسیلی که محصول میتواند در آینده داشته باشد و همجنین چالش هایی که در آینده ممکن است برای محصول پیش بیاید را بررسی میکند و طبق آن میتوان تصمیم گرفت که برای رسیدن به آن اهداف تعیین شده باید چه کار هایی را انجام داد.	
	
	
حال برای نوشتن این چشم انداز محصول باید موارد زیر را بررسی بکنیم:

	
\subsection{\textbf{مشتریان و کاربران هدف}} \label{section.productVision.customers}


طبیعتا در حال حاضر تعداد فروشگاه ها و تنوع فروشگاه هایی که با آن ها قرارداد بسته شده اند کم و محدود هستند. از جمله کار هایی که باید در آینده انجام داد این است که تنوع فروشگاه ها افزایش یابد و قرارداد با انواع مختلف فروشگاه ها مانند فروش کاپشن، انواع شلوار، کت و شلوار مجلسی، لباس های \lr{casual}بیرونی، لباس های اسپرت و ... بسته شود تا بتوان تعداد کاربرانی که شبکه کسب و کار ما آن را پوشش میدهد بیشتر شود و در واقع بازار هدف خود را بزرگتر کنیم.


همچنین برای گسترش هرچه بیشتر بازار هدف باید مناطق جغرافیایی تحت پوشش را گسترش داد و از تمامی استان ها و شهر های بزرگشان چندین مغازه را تحت پوشش قرار داد و با آن ها قرارداد بست.


اینکار علاوه بر اینکه مدت زمان رسیدن یکسری از سفارشات خاص را سریع تر میکند همچنین باعث میشود که سلایق افراد مختلف استان ها نیز همگی در یک مکان قرار گیرند و ممکن است به صورت تساعدی تعداد سفارشات لباس افزایش یابد.



\pagebreak

\section{\textbf{گزارش نحوه انجام پروژه}} \label{section.report}

در ابتدا با توجه به امتحان میانترم، برنامه ریزی فاز اول پروژه را تا یک روز بعد از امتحان به تعویق انداختیم سپس برنامه ریزی انجام شد و هر یک از مسائل به یکی از اعضا سپرده شد. بنا بر این بود که بعد از اینکه هر یک از اعضا مسئله ی مربوط به خود را در قالب فایل ورد در گیت هاب به اشتراک گذاشتند، فایل گزارش در لاتک توسط \underline{مهدی محسنی} نوشته شود و ساخت \lr{Burn Down Chart} به \underline{محراب کشاورز گیلده }سپرده شد. همچنین ددلاین نوشتن مسائل تا روز دوشنبه 8 دی که آخرین روز \lr{sprint} است، تعیین شد و نوشتن فایل گزارش با لاتک نیز به روز چهارشنبه موکول شد. این تاخیر به دلیل آشنایی نداشتن با لاتک و همین طور امتیازی بودن نمره ی آن انجام شد.

خوشبختانه تمامی اعضای گروه مطابق با ددلاین تعیین شده، وظایف خود را انجام دادند. فقط یک نکته اتفاق افتاد که اعضا با توجه به تازه کاری در ارتباط با گیت هاب بعد از انجام وظیفه ی خود، فقط آن را از قسمت \lr{To Do} به قسمت \lr{Done} انتقال دادند و گزینه ی \lr{Close Issue} را نزدند! برای همین تمام وظایف در روز چهارشنبه 10 دی بسته شد.




\end{document}