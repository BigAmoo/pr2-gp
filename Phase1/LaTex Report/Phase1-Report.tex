%preamble
\documentclass[12pt,a4paper]{article}

%inputs for firstpage
\def \CourseName {درس \lr{MIS}}
\def \Instructor {دکتر مسلم حبیبی}
\def \Semester {نیم‌سال دوم\\سال تحصیلی 99-00}


%packages
\usepackage[]{algorithm2e}
\usepackage{cite}
\usepackage{calc}
\usepackage{fancyhdr}
\usepackage{lipsum}
\usepackage{color}
\usepackage{ragged2e}
\usepackage[inline]{enumitem}
\usepackage[dvipsnames]{xcolor}
\usepackage{graphicx}
\usepackage{wrapfig}
\usepackage{float}
\usepackage[skip=12pt,indent=2em]{parskip}
\usepackage{setspace}
\usepackage{textcomp}
\usepackage{etoolbox}
\usepackage{xpatch}
\usepackage{tabu}
\usepackage{hyperref}


%for persian fonts
\usepackage{xepersian}
\defpersianfont\bnazanin{BNazanin}
\settextfont{BNazanin}

\title{
	\center
	\includegraphics[width=4cm, height=4cm]{images/shariflogo.jpg} 
	\small 
	\\ دانشگاه صنعتی شریف\\دانشکده مهندسی صنایع \\ درس \lr{MIS}
	\normalsize
	\\[60pt]
	\textbf{\huge\CourseName}
}
\author{
	\textbf{استاد درس:}
	\\
	\Instructor
 	\\
	\\
	\textbf{نام اعضای گروه:}
	\\مهدی محسنی 
	\\محراب کشاورز گیلده 
	\\احسان چشمی
	\\[45pt]
}
\date{{\small\Semester}}



%body
\begin{document}
 
 \maketitle
 

	
	
	\section{\lr{\textbf{Product Vision}}} \label {section.productVision}
	چشم انداز محصول توصیف کننده هدف یک محصول است که با چه قصدی آن را ایجاد کرده ایم و به چه اهدافی در رابطه با مشتریان خود و سایر افراد مرتبط با کسب و کار میخواهیم برسیم. چشم انداز محصول وضعیت آینده محصول را توصیف میکند و پتانسیلی که محصول میتواند در آینده داشته باشد و همجنین چالش هایی که در آینده ممکن است برای محصول پیش بیاید را بررسی میکند و طبق آن میتوان تصمیم گرفت که برای رسیدن به آن اهداف تعیین شده باید چه کار هایی را انجام داد.	
	
	
حال برای نوشتن این چشم انداز محصول باید موارد زیر را بررسی بکنیم:

	
\textbf{مشتریان و کاربران هدف:}


طبیعتا در حال حاضر تعداد فروشگاه ها و تنوع فروشگاه هایی که با آن ها قرارداد بسته شده اند کم و محدود هستند. از جمله کار هایی که باید در آینده انجام داد این است که تنوع فروشگاه ها افزایش یابد و قرارداد با انواع مختلف فروشگاه ها مانند فروش کاپشن، انواع شلوار، کت و شلوار مجلسی، لباس های \lr{casual}بیرونی، لباس های اسپرت و ... بسته شود تا بتوان تعداد کاربرانی که شبکه کسب و کار ما آن را پوشش میدهد بیشتر شود و در واقع بازار هدف خود را بزرگتر کنیم.


همچنین برای گسترش هرچه بیشتر بازار هدف باید مناطق جغرافیایی تحت پوشش را گسترش داد و از تمامی استان ها و شهر های بزرگشان چندین مغازه را تحت پوشش قرار داد و با آن ها قرارداد بست.


اینکار علاوه بر اینکه مدت زمان رسیدن یکسری از سفارشات خاص را سریع تر میکند همچنین باعث میشود که سلایق افراد مختلف استان ها نیز همگی در یک مکان قرار گیرند و ممکن است به صورت تساعدی تعداد سفارشات لباس افزایش یابد.


\textbf{حوزه های ارزش ذی نفعان:}
	ذی نفعان ما شامل افراد و گروه های زیر میشود:
	\textbf{مشتریان: }سه ارزش مهم برای مشتری وجود دارد قیمت، کیفیت، تنوع. برای قیمت محصولات دست کسب و کار ما کمی بسته است چون بسیار بستگی به خود فروشگاه ها دارد که آیا بخواهند از استراتژی تخفیف برای محصولات خود استفاده بکنند یا نه.
	
	
	در مورد کیفیت محصولات میتوان از جمله کار هایی که انجام داد درخواست اطلاعات کامل تر و دقیق تر از محصولات از فروشگاه ها کرد تا دیتا هایی که در سایت قرار میگیرند کامل تر باشند و همچنین تعداد عکس های موجود نیز بیشتر باشد. علاوه بر آن مشخصات سایز بندی دقیق تر باشند و به طور مثال علاوه بر سایز طول و عرض لباس نیز نوشته شود تا انتخاب محصول برای مشتری راحت تر باشد.
	
	
	برای افزایش میزان تنوع نیز باید همان مسیری که در قیمت مشتریان و کاربران هدف ذکر شده است را پیش گرفت و با افزایش تعداد قرارداد ها با فروشگاه های مختلف تنوع را زیاد تر کرد.
	
	
	\textbf{فروشگاه ها: }تضمین دهی به فروشگاهیان که مشتریان ما پرداخت هایشان کامل و درست بوده است (مخصوصا در آمریکا که با کردیت خرید انجام میشود). همچنین باید سیستم پیک به نحوی خوب باشد که در هنگام فرستادن محصول به دست مشتریان آسیبی به آن وارد نشود و یا اگر مشتری لباسی را خواست پست بدهد که مشکلی نداشته و صرفا از انتخاب خود ناراضی بوده است، پیک ها باید موظف شوند که محصول را کاملا سالم به دست فروشگاه ها برسانند.
	
	
	علاوه بر آن میتوان یک سری طرح های خاص را برای فروشگاه ها ارائه کرد به طور مثال تخفیفات خاص را به آنها پیشنهاد داد که در روز های خاص در سایت نمایش داده شوند و باعث شود که هماهنگی بین فروشگاه ها نیز بیشتر شود. به طور مثال یک جشنواره ای سایت ما برگزار میکند برای بهتر شدن و بهتر عملکردن این جشنواره باید با فروشگاه ها هماهنگ کرد که تخفیفاتشان را به خوبی ارائه دهند و نه تخفیف کم داده باشند و نه تخفیف زیاد تا هم آن ها از سود خود ضرر نکنند و هم تعداد فروش ها در آن جشنواره کم نباشد.
	
	
\textbf{پیک: }یکی از مسائل مهم بحث رسیدن محصول به دست مشتریان و همچنین پس دادن محصولات بد تحویل گرفته شده است. در حال حاضر تا حد خوبی این موارد پوشش داده شده اند. اما در آینده از جمله کار هایی که میتوان انجام داد سیستم تحویل دهی محصولات و تحویل گیری آن ها است که بسیار بستگی به تکنولوژی های فراگیر در آن زمان دارد. به طور مثال اگر فرض کنیم این کسب و کار در یک کشور پیشرفته مثل امریکا بوده است بر اساس میزان تقاضای مشتریان میتوان تصمیم گرفت که آیا از پهباد برای فرستادن سفارشات استفاده کرد یا نه.


	همچنین علاوه بر آن باید مکان هایی که پیک های مختلف قرار میگیرند را افزایش داد تا سرعت رسیدن به فروشگاه ها افزایش یابد و سریع تر بتوان محصول را به دست مشتری رساند.
	
	
	\textbf{ویژگی های اصلی و منحصر به فروش محصول:}
	از جمله ویژگی های بسیار مهم و منحصر به فروش لباس ها میزان دیتا های نشان داده شده از آن ها در سایت است. بدین منظور باید یک چارچوب کلی و دقیق برای تمامی فروشگاه هایی که با آن ها قرارداد داریم فرستاده شود تا بر اساس آن چارچوب اطلاعات محصول خود را به ما ارائه دهند و بتوان هر چه بهتر آن دیتا ها را به نمایش مشتریان و کاربران خودمان بگذاریم. در این بین اگر حجم این دیتا ها برای هر محصول زیاد شود باید \lr{user friendly} و کاربر پسند بودن محیط کاربری سایت را نیز مد نظر قرار داد که این اطلاعات به صورت سطح بندی شده به نمایش گذاشته شوند به طور مثال در ابتدا فق طاطلاعات مهم مثل قیمت و رنگ و سایر محصول نمایش داده شود سپس در قسمت اطلاعات بیشتر به صورت صفحه بندی شده قسمت های مختلف بررسی شوند و مثلا اطلاعاتی مانند اندازه دقیق لباس، جنس آن، ویژگی های خاص (مثلا تعداد جیب های یک شلوار) آن قرار داده شود.
	همچنین علاوه بر مشخصاتی که هر جنس محصول دارد باید به تنوع و بخش بندی های کلی خود سایت نیز به خوبی پرداخته شود. چون کل پایه کسب و کار بر اساس سایت است باید به خوبی به ویژگی های ریز آن پرداخت حتی در مواقعی رنگ پست زمینه استفاده شده در سایت نیز بررسی میشود که زننده نباشد و باعث خستگی افراد نشود.
	بخش بندی های سایت باید به خوبی انجام شوند و کاملا به صورت لیست بندی شده با فیلتر های گوناگون محصولات نمایش داده شوند. به طور مثال فیلتر بر اساس نوع محصول (شلوار، پیراهن، لباس راحتی و ...) و یا فیلتر بر اساس سن و جنسیت افراد (دختر، پسر، زن، مرد، افراد مسن) همچنین باید در کنار این فیلتر ها امکان فیلتر کردن بر اساس ویژکی های محصول نیز قرار داده شود به طور مثال روی قیمت یک دسته محصولی که مشتری در حال مشاهده آن است فیلتر انجام شود یا روی فروشگاه ارائه دهنده آن یا حتی جنس لباس و کارخانه تولیدی آن و اطلاعاتی از این قبیل.
	
	
\section{\lr{User Story}} \label{section.userStory}	



\textbf{داستان کاربری مشتری:}\\
1.	به عنوان مشتری نیاز دارم هنگام ثبت نام حساب، اگر خطایی مرتکب می شوم به وضوح مطلع شوم تا بتوانم سریع آن را برطرف کنم.\\
2.	به عنوان مشتری ، باید بتوانم وارد حساب کاربری خود بشوم تا بتوانم جزئیات حساب و سفارشات گذشته خود را ببینم.\\
3.	به عنوان مشتری که به دنبال کالایی هستم، اگر کالایی در فروش است باید بتوانم از آن مطلع شوم تا ترغیب به خرید آن شوم.\\
4.	به عنوان مشتری که به دنبال کالایی هستم، باید مناسب ترین گزینه ها برای من نمایش داده شود تا با احتمال بیشتری آنچه را که می خواهم پیدا کنم.\\
5.	به عنوان مشتری باید از برنامه مزایای وفاداری مطلع شوم تا از این برنامه بهره ببرم.\\
6.	به عنوان مشتری در شرف خرید، باید بتوانم مشخصات کارت اعتباری خود را ارائه و احراز هویت شوم تا بتوانم پرداخت خود را تکمیل کنم.\\
7.	به عنوان مشتری در شرف خرید، نیاز دارم در صورت عدم موفقیت پرداخت سبد خرید خود را بازیابی کنم تا در وقتم صرفه جوی کنم.\\
8.	به عنوان مشتری که خریدی انجام داده، باید سفارش من به موقع و به درستی تکمیل شود تا مجبور نشوم از طریق خدمات مشتری پیگیری کنم.\\
9.	به عنوان مشتری که خریدی انجام داده، باید از شرایط مرجوعی مطلع شوم تا در صورت نیاز بدون مشکل مراحل را طی کنم.


داستان کاربری صاحب فروشگاه
1.	به عنوان صاحب فروشگاه باید قبل از مراجعه از شرایط عقد قرارداد مطلع شوم تا مشکلی در مراحل ثبت نام پیش نیاید و نیازی به مراجعه مجدد نباشد.
2.	به عنوان صاحب فروشگاه باید بتوانم وارد حساب کاربری خود بشوم تا بتوانم جزئیات حساب و معاملات گذشته خود را ببینم.
3.	به عنوان صاحب فروشگاه باید توانایی اضافه، حذف و ویرایش لیست کالاها در هر لحظه را از طریق حساب کاربری داشته باشم.
4.	به عنوان صاحب فروشگاه باید بتوانم امتیاز و نظرات مشتریان را ببینم تا درکی از عملکرد خود به دست آورم.
5.	به عنوان صاحب فروشگاه باید از شرایط تبلیغات و پروموت کردن محصولات خود در سامانه مطلع شوم تا بتوانم مشتریان بیشتری جذب کنم.
6.	به عنوان صاحب فروشگاه که از طریق سیستم خریدی از او شده، باید در اسرع وقت از خرید مطلع شوم تا بتوانم کالا را در زمان مناسب به پیک تحویل بدهم.
7.	به عنوان صاحب فروشگاهی که خریدی از او تکمیل شده، بعد از فروش کالا باید در اسرع وقت مبلغ فروش به حسابم واریز شود تا دچار زیان نشوم.
8.	به عنوان صاحب فروشگاهی که کالایی از او مرجوع شده، باید از دلایل آن مطلع شوم تا از تکرار آن جلوگیری کنم.
9.	به عنوان صاحب فروشگاهی که کالایی از او مرجوع شده، باید بتوانم مشخصات کارت اعتباری خود را ارائه و احراز هویت شوم تا هزینه کالای مرجوعی را بازگردانم.

داستان کاربری پیک
1.	به عنوان پیک باید قبل از مراجعه از شرایط عقد قرارداد مطلع شوم تا مشکلی در مراحل ثبت نام پیش نیاید و نیازی به مراجعه مجدد نباشد.
2.	به عنوان پیک باید بتوانم وارد حساب کاربری خود بشوم تا بتوانم جزئیات حساب و ماموریت های گذشته خود را ببینم.
3.	به عنوان پیک باید بتوانم امتیاز و نظرات مشتریان را ببینم تا درکی از عملکرد خود به دست آورم.
4.	به عنوان پیکی که ماموریتی برای او ارسال شده، باید از جزییات ماموریت قبل از قبول کردن آن مطلع شوم تا بتوانم در مورد پذیرفتن آن تصمیم گیری کنم.
5.	به عنوان پیکی که ماموریتی را پذیرفته، نیاز دارم که از قبل به صاحب فروشگاه و خریدار در مورد زمانبندی رسیدن پیک اطلاع داده شود تا بدون معطلی ماموریت خود را تکمیل کنم.
6.	به عنوان پیکی که ماموریتی را تکمیل کرده باید هزینه آن را در اسرع وقت دریافت کنم تا دچار زیان نشوم.

\section{\lr{Functional Requirement}} \label{section.functionalRequiremets}	
	\textbf{نیازمندی های کاربردی} \\
	•	به نیازمندی هایی گفته می شود که مشتری آن ها را صراحتا از ما می خواهد. در این پروژه این نیازمندی ها عبارتند از:
	•	ثبت نام و تکمیل مشخصات فرد با شماره همراه یا ایمیل
	•	ارسال پیامک یا ایمیل برای تایید ثبت نام
	•	ثبت نام حضوری صاحبان فروشگاه ها
	•	صاحبان فروشگاه ها صفحه شخصی داشته باشند که درون آن اطلاعات فروشگاه باشد و صاحب فروشگاه قدرت اضافه و حذف و ویرایش لیست کالاها به همراه مشخصاتشان را داشته باشد.
	•	ثبت نام حضوری پیک های موتوری و ثبت مشخصات خود و وسیله نقلیه شان
	•	ردیابی پیک های موتوری به کمک گوشی هوشمند
	•	نشان دادن لیستی از فروشگاه های لباس نزدیک به هر کاربر و هزینه پیک به او
	•	انتخاب فروشگاه توسط کاربر
	•	مشاهده لیست کالاهای موجود فروشگاه انتخاب شده توسط کاربر به او
	•	انتخاب کالاهای مورد نیاز و تعداد هر کدام توسط کاربر و اضافه شدن این کالا ها به سبد خرید او
	•	بررسی عدم مشکل در کالاهای انتخابی توسط کاربر (وجود کالا و سفارش در محدوده زمانی درست)
	•	ارسال سفارش توسط کاربران
	•	پرداخت هزینه توسط کاربر از طریق درگاه بانکی
	•	ارسال پیام موفقیت پرداخت در صورت موفقیت به کاربر
	•	متوقف شدن عملیات خرید و نگهداشته شدن لیست خرید کاربر در صورت عدم موفقیت پرداخت
	•	در صورت موفقیت پرداخت لیست خرید به سامانه ارسال شود
	•	لیست سفارشات مشتری به صاحب فروشگاه داده شود
	•	جستجو روی پیک های موتوری آنلاین و در دسترس 
	•	انتخاب نزدیک ترین موتور برای ارسال سفارش
	•	تایید پیک برای قبول سفارش
	•	ارسال اطلاعات خرید (آدرس فروشگاه، آدرس مقصد و لیست خرید) به پیک موتوری
	•	امکان رد سفارش توسط پیک
	•	جستجوی دوباره و پیدا کردن پیکی دیگر در صورت رد سفارش توسط پیک (این روند تا زمان پیدا شدن پیک ادامه می یابد)
	•	امتیاز دهی به فروشگاه و محصول و پیک و ذخیره شدن آن ها
	•	امکان مرجوع کردن کالای خریداری شده در صورت خواست مشتری
	•	تماس کاربر با تیم پشتیبانی و در صورت تایید ارسال پیک موتوری برای بازگرداندن کالا
	محدودیت ها 
	•	ممکن است سایز لباس ها برای افراد مناسب نباشد.
	•	امکان ست کردن چند لباس با هم وجود ندارد.
	•	امکان تست کردن لباس های مختلف بدون سفارش دادن آن ها برای مشتریان وجود ندارد.
	•	فروشنده قدرت توضیح دادن و تعریف کردن از محصول را ندارد.
	•	مشتری قدرت دست زدن و دیدن جنس محصول از نزدیک را ندارد.
	•	امکان این وجود ندارد که از فروشنده بپرسیم چه جنسی می خواهیم و فروشنده ما را راهنمایی کند.
	•	امکان نسیه بردن جنس وجود ندارد.
	•	اگر فروشنده جنسی را نداشته باشد و بخواهد به زودی آن را بیاورد امکان این را ندارد که به مشتری این اطلاع را بدهد.
	
	
\end{document}